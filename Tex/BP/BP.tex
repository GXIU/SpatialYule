\documentclass[reprint,unsortedaddress,amsmath,amssymb,floatfix,aps,prl,showkeys]{revtex4-2}
\usepackage{graphicx}% Include figure files
\usepackage{dcolumn}% Align table columns on decimal point
\usepackage{subfigure}
\usepackage{bookmark}
% \usepackage{biblatex}
\usepackage{float}
\usepackage{url}
\usepackage{bm}% bold math
\usepackage{hyperref}% add hypertext capabilities
\usepackage[mathlines]{lineno}% Enable numbering of text and display math

\begin{document}
\title{Geographical Boundary Problems}
\author{Gezhi Xiu, Jianying Wang, Lei Dong}
\author{Yu Liu}
\email{liuyu@urban.pku.edu.cn}
\affiliation{Institute of Remote Sensing and Geographic Information System (IRSGIS), Peking University}
\date{\today}

\begin{abstract}
    City is a kind of complex systems that grows organically and expandingly, thus treating cities as a potential system is a reasonable way to interpret them. Boundary problem is a general kind of mathematical description for the systems that overall status can be determined from the status at the boundary. In this work, we establish a general model for urban development problem by considering the urban develop by investigating the growth behaviour at the edges of cities. We derive statistical tests in principle of Diriclet and Green problems and conduct numerical tests for multiple cities. We find that urban growth dynamic can be well approximated by edging growth.
\end{abstract}

\maketitle

\begin{align*}
\begin{array}{ccc}
    \frac{{\mathrm{d}}x}{{\mathrm{d}}t} = c\left[\right.(b+y)(1-k)s{(1-y)}^{a}\\ -x\left((1-k)(1-s){(1-x)}^{a}+k(1-s){(1-x)}^{a}\right)\left]\right.,\\ \frac{{\mathrm{d}}y}{{\mathrm{d}}t} = c\left[\right.(b+x)(1-k)(1-s){(1-x)}^{a}\\ -y\left((1-k)s{(1-y)}^{a}+ks{(1-y)}^{a}\right)\left]\right..
\end{array}\\
\begin{array}{ccc}\frac{{\mathrm{d}}x}{{\mathrm{d}}t} = c\left[\right.(b+y)(1-k)s{(1-y)}^{a}\\ -x\left((1-k)(1-s){(1-x)}^{a}+k(1-s){(1-x)}^{a}\right)\left]\right.,\\ \frac{{\mathrm{d}}y}{{\mathrm{d}}t} = c\left[\right.(b+x)(1-k)(1-s){(1-x)}^{a}\\ -y\left((1-k)s{(1-y)}^{a}+ks{(1-y)}^{a}\right)\left]\right..\end{array}
\end{align*}


% \bibliography{ref.bib}
\end{document}