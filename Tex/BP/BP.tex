\documentclass[reprint,unsortedaddress,amsmath,amssymb,floatfix,aps,prl,showkeys]{revtex4-2}
\usepackage{graphicx}% Include figure files
\usepackage{dcolumn}% Align table columns on decimal point
\usepackage{subfigure}
\usepackage{bookmark}
% \usepackage{biblatex}
\usepackage{float}
\usepackage{url}
\usepackage{bm}% bold math
\usepackage{hyperref}% add hypertext capabilities
\usepackage[mathlines]{lineno}% Enable numbering of text and display math

\begin{document}
\title{Geographical Boundary Problems}
\author{Gezhi Xiu, Jianying Wang, Lei Dong}
\author{Yu Liu}
\email{liuyu@urban.pku.edu.cn}
\affiliation{Institute of Remote Sensing and Geographic Information System (IRSGIS), Peking University}
\date{\today}

\begin{abstract}
    City is a kind of complex systems that grows organically and 
    Boundary problem is a general kind of mathematical description for the systems that overall status can be determined from the status at the boundary. 
\end{abstract}

\maketitle



% \bibliography{ref.bib}
\end{document}