\documentclass[reprint,unsortedaddress,amsmath,amssymb,aps,prl,showkeys]{revtex4-2}
\usepackage{graphicx}% Include figure files
\usepackage{dcolumn}% Align table columns on decimal point
\usepackage{subfigure}
\usepackage{bookmark}
% \usepackage{biblatex}
\usepackage{float}
\usepackage{url}
\usepackage{bm}% bold math
\usepackage{hyperref}% add hypertext capabilities
\usepackage[mathlines]{lineno}% Enable numbering of text and display math

\begin{document}
\title{Memory Matters for Citeis}
\author{Gezhi Xiu, Jianying Wang, Lei Dong}
\author{Yu Liu}
\email{liuyu@urban.pku.edu.cn}
\affiliation{Institute of Remote Sensing and Geographic Information System (IRSGIS), Peking University}
\date{\today}

\begin{abstract}
    Fast urbanization process leads to unforeseen dilemmas. Stately the secondary phase of urban growth, rather than naturally attracting people elsewhere, exhibits restricted dynamics for internal conflicts. Here, we propose a model to reproduce the statistical features of growing urban systems, such as population density within a city (Clark's law), fractality, and cities' rank size distributions (Zipf's law). We additionally add a \emph{memory kernel} to account for the development ceiling of the region. We show that cities exhibit transitions in various ways such as the shuffling of city ranks and urban shrinking, suggesting the overall resource as the upper bound for later stage of regional growth of urbanization.
\end{abstract}

\maketitle

Urban growth dynamics has always drawn dramatic attention in the past few decades. This problem lies in twso parts, population settlement and economic growth, both of which changes spatially over time. Traditional studies of spatial economics have attempted to construct this phenomenon under equilibrium models of regional urban systems\cite{batty1992form}. These models base their idea on agglomeration economies to explain why urban features tend to gather. However, the dynamics of shifts between urban equilibriums (say, polycentric transitions of cities), is poorly interpreted throughout these models. First, the initial states of urban structure are mostly isotropic, that the present state of cities may be only one possible state of urban equilibriums. Moreover, urban constructions, especially urban sprawling and urban shrinkage, show that cities are out-of equilibrium systems with unsynchronized growth between infrastructure and population distribution. Second, these models mostly address city agglomeration's benefits by the interaction between pairs of individuals. This brings difficulties in understanding the hierarchy and functional divisions that can be seen in nearly all cities. Yet, separating the growth dynamics of infrastructures and population has not yet been taken serious attention in existing models, though some work in complex networks has realized that preferential attachment has hysteresis effect in building contacts. Lastly, throughout the abundant data describing cities, existing models cannot make quantitative predictions with the more information added. We present in this Letter a stochastic growing model of urban systems, which relies on the assumption that the agglomeration effects of infrastructures and population are separated, which can cause the urban core to shift over time. 

Following the discrimination of growing preferences of population and infrastructures, we omit certain details and focus on the basic growth process over longer period of time. We thereby build a minimal version of model that replicates the basic properties of urban systems and is able to account for the unforeseen cases both qualitatively and quantitatively. The model we propose is thus the essence of the evolution of cities' mass as population grows. We focus on newly joint population's pulling the force of infrastructure building, and its impact on urban shrinkage and regional rank injection. 

We first consider the process of urbanization on individual level. People either form new cities, or join the existing cities. We assume that each of these process happens at a constant rate, that is, the times of forming new cities per unit time is proportional to the existing cities, and the population of newly joint citizen of each cities is proportional to the cities' population. We can denote the two proportions as $\beta_1$ and $\beta_2$. These settings are parallel to U. Yule's original model\cite{yule1925ii} based on the observation of older genus and species tend to have existed longer. This observation also stands for urban systems under stable socio-economic conditions. % This model can reformulate Zipf's law of rank size distribution of cities, as long as $\beta_1$ approximates $\beta_2$.

In dealing with sptial transition of urban structures, we set up the locating rules for urban population. Assuming that the considered area is in $L\times L$ continuous space with grids (representing urban blocks), corresponding to possessiveness: each grid is owned by the city that governs the first citizen who lands on it. Speaking of two parts of growing process, the locating procedures are defined as follows. The city emerges at a randomly chosen, but untaken place (or the city fails to be established). In every emerged city, a newly joint citizen is \emph{introduced} by a existing citizen, so that she is located at a distance of $r$ near the existing one of a random direction $\theta$. We take $r\lesssim 1$ to make sure that new comer is located at or at the neighbor of inviter's block. In global perspective, cities growing process is like a diffusion process with some seed points sprinkled on the area as grids are gradually divided into the regime of growing cities. We name this process as \emph{Spatial Yule Process (SYM)}. As we can see, blocks with more nodes have higher probability to introduce new people nearby. Such natural formation of urban systems resembles the population distribution in emergence of many regional systems with adjustable parameters. But regional growth also face difficulties of economical bottlenets and the need of balancing regional growths. In the meantime, the regional authority can only devote limited amount of total resource in infrastructure construction. These facts imply that we have to add a limit on growing process to reflect the truth. Thus we introduce a mechanism called the \emph{memory kernel}. It can be interpreted by the following statement: New comers needs supply to settle, thus he carries a \emph{coin} to get her settled. However, the total regional resource is limited that the amount of coin is limited, say $N^*$. So from the $N^*+1$ person that settles in a city of this region, she carries a coin as a pre-comer loses hers. Thus when the regional fortune has all been shared, all money is only transfered to the new comers, to allow their essential needs of infrastructures. 

We now discuss the variables $\beta:=\beta_1/\beta_2$ and $r$. The problem of determining the relative speed of city generation is very reminiscent of some problems encountered in gas physics. It is interesting to investigate the number of cities in a given regions of the same population. Some groups tend to form new cities to have sufficient infrastructures and less diversity of urban output ($\beta\gtrsim 1$) and some cities may go otherwise ($\beta\lesssim 1$). This value is actually a reflection of the intensity of regional industry. Speaking of $r$, the metropolis areas over the world have very different densities. In SYM, it determines the sprawl of a city with given population. It can also be taken as the area proportion for a city in the studied region. On the other hand, it is also constraint of regional growth controlling the expected allowance of cities. We take $r=1/2$ as default simulating condition in this Letter.

As for the size of the memory kernel, $N^*$, we realize that it is related to the authority's financial ability to be supportive of more incomer of the region.

In summary, our model is defined under three parameters, $\beta$, $r$, and $N^*$. In every exponential time (before the memory kernel is filled), an existing citizen introduces a new individual located from $(r,\theta)$ of her. Meanwhile a new city is introduced by an existing city in another exponential time at somewhere empty. So that the urban system is growing acceleratingly at the first phase, during which every individual owns a coin with her. In the second phase where the memory kernel is full, i.e. the total population of the region is over $N^*$, only those who own coins (summing up to $N^*$) can introduce new comers near to them in the future. Existing ones lose their ability to introduce gradually and almost surely.

% Cities are places that concentrate human innovations and productivities. It has been well studied that urban output grows as fast as urban sizes. This results in some interesting results of urban studies on various scales. For inner-city properties, the classical common knowledge is that the population density decay exponentially from core to periphery; As well as cross-city level, the rank size distribution of cities, known as Zipf's law, is the indicator for how much more attractive a bigger city is than smaller cities. These findings are keys to understanding urban formation patterns. In recent years, researchers in urban studies are devoted to reformulate complex cities with simple rules. 

% While our interest lies in hte spatial distribution of population, there are other feathers of urban growth for which parallels with economical development over geographical space can be drawn, for mathematical methods applied. For example, the rank-frequency distribution of urban population\cite{gabaix1999zipf's}, can be explained using a novel form of the Yule process, first introduced to explain the distribution of the number of species

\bibliography{ref.bib}
\end{document}
