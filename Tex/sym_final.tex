\documentclass[reprint,unsortedaddress,amsmath,amssymb,aps,prl,showkeys]{revtex4-2}
\usepackage{graphicx}% Include figure files
\usepackage{dcolumn}% Align table columns on decimal point
\usepackage{subfigure}
\usepackage{bookmark}
% \usepackage{biblatex}
\usepackage{float}
\usepackage{url}
\usepackage{bm}% bold math
\usepackage{hyperref}% add hypertext capabilities
\usepackage[mathlines]{lineno}% Enable numbering of text and display math

\begin{document}
\title{Memory Matters for Citeis}
\author{Gezhi Xiu, Jianying Wang, Lei Dong}
\author{Yu Liu}
\email{liuyu@urban.pku.edu.cn}
\affiliation{Institute of Remote Sensing and Geographic Information System (IRSGIS), Peking University}
\date{\today}

\begin{abstract}
    Fast urbanization process leads to unforeseen dilemmas. Stately the secondary phase of urban growth, rather than naturally attracting people elsewhere, exhibits restricted dynamics for internal conflicts. Here, we propose a model to reproduce the statistical features of growing urban systems, such as population density within a city (Clark's law), fractality, and cities' rank size distributions (Zipf's law). We additionally add a \emph{memory kernel} to account for the development ceiling of the region. We show that cities exhibit transitions in various ways such as the shuffling of city ranks and urban shrinking, suggesting the overall resource as the upper bound for later stage of regional growth of urbanization.
\end{abstract}

\maketitle

Urban growth dynamics has always drawn dramatic attention in the past few decades. This problem lies in two parts, population settlement and economic growth, both of which changes spatially over time. Traditional studies of spatial economics have attempted to construct this phenomenon under equilibrium models of regional urban systems\cite{batty1992form}. These models base their idea on agglomeration economies to explain why urban features tend to gather. However, the dynamics of shifts between urban equilibriums (say, polycentric transitions of cities), is poorly interpreted throughout these models. First, the initial states of urban structure are mostly isotropic 

% Cities are places that concentrate human innovations and productivities. It has been well studied that urban output grows as fast as urban sizes. This results in some interesting results of urban studies on various scales. For inner-city properties, the classical common knowledge is that the population density decay exponentially from core to periphery; As well as cross-city level, the rank size distribution of cities, known as Zipf's law, is the indicator for how much more attractive a bigger city is than smaller cities. These findings are keys to understanding urban formation patterns. In recent years, researchers in urban studies are devoted to reformulate complex cities with simple rules. 

While our interest lies in hte spatial distribution of population, there are other feathers of urban growth for which parallels with economical development over geographical space can be drawn, for mathematical methods applied. For example, the rank-frequency distribution of urban population\cite{gabaix1999zipf's}, can be explained using a novel form of the Yule process, first introduced to explain the distribution of the number of species

\bibliography{ref.bib}
\end{document}
