\documentclass[reprint,unsortedaddress,amsmath,amssymb,aps,prl,showkeys]{revtex4-2}
\usepackage{graphicx}% Include figure files
\usepackage{dcolumn}% Align table columns on decimal point
\usepackage{subfigure}
\usepackage{bookmark}
% \usepackage{biblatex}
\usepackage{float}
\usepackage{url}
\usepackage{bm}% bold math
\usepackage{hyperref}% add hypertext capabilities
\usepackage[mathlines]{lineno}% Enable numbering of text and display math

\begin{document}
\title{The Dynamical Distribution of Urban Growth}
\author{Gezhi Xiu, Jianying Wang, Lei Dong}
\author{Yu Liu}
\email{liuyu@urban.pku.edu.cn}
\affiliation{Institute of Remote Sensing and Geographic Information Systems (IRSGIS), Peking University}
\date{\today}

\begin{abstract}
    Empirical evidence suggests that the evolution of urban systems is not only determined by local conditions, but also is constrained by regional status. We propose an out-of-equilibrium model of emerging cities within a given region, which explains the spatial transitions of developmental focus and urban shrinkage phenomenon in developed cities. Meanwhile the model analytically keeps the classical results such as Clark's law for urban population density, and Zipf's law for cities' rank size distributions. We derive that these classical properties are valid for developing areas, or equivalently, most of the present cities; and the second phase of our model predicts the inevitability of various urban diseases given the limited regional resource. 
\end{abstract}

\maketitle

\section{Introduction}

The spatial and democratic growth of cities resembles the two-dimensional aggregation of particles, which has led in a trend to model urban growth through statistical physics, specifically, cluster formation by the proliferation of development units. However, the cross-scale properties of complex cities such as regional vicissitude, urban shrinkage, and the spatial transition of urban cores, are hard to reformulate by existing models. Here we propose a spatial growth model, in which development units are limited by amount, rather than added permanently, is better able to capture the non-equilibrium dynamics of cities in an urban system.




\end{document}