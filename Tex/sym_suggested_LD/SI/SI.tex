\documentclass[aps,prl]{revtex4-2}
\usepackage{graphicx,amssymb,amsmath}
\bibliographystyle{apsrev4-2}
\renewcommand*{\citenumfont}[1]{S#1}
\renewcommand*{\bibnumfmt}[1]{[S#1]}


\begin{document}


\title{Supplementary Material\\Memory Matters for Cities}
\date{\today}
\author{Gezhi Xiu, Jianying Wang, Lei Dong}
% \email{xiugz@pku.edu.cn}
% \affiliation{IRSGIS, Peking University}

\author{Yu Liu}
\email{liuyu@urban.pku.edu.cn}
\affiliation{IRSGIS, Peking University, Beijing, China}
% \altaffiliation{CAMS (CNRS/EHESS) 190-198, avenue de France, 75244 Paris Cedex 13, France}

\pacs{} 

% If your reference list includes text notes as well as references,
% include the following line; otherwise, comment it out.


\maketitle
\tableofcontents
\vspace{1cm}

\section{Mathematical Proofs}

\subsection{Average Age within the Memory Kernel}

We consider the active population over the whole region as working population. The generation speed of population is $N^*\beta_2 + k\beta_1$. Thus the population is refreshing, leading to a constant expected age of \[N^*/\beta'\] for $\beta' := (N^*\beta_2 + k\beta_1)/N^*+k$. This leads to a practical implication that the working age allowed in cities shall be related with the sum of working offer from central government of a region. For instance, in the United States, the regressive value of $\beta$ is around $0.04$. We assume that the working years of a person is 40 years, the expected $N^*$ for the United States is 1,600 units, i.e., 4 millions distributive working opportunities in all cities.




% \bibliographystyle{plain}
\bibliography{refs.bib}
    
\end{document}