\documentclass[reprint,unsortedaddress,amsmath,amssymb,aps,prl,showkeys]{revtex4-2}
\usepackage{graphicx}% Include figure files
\usepackage{dcolumn}% Align table columns on decimal point
\usepackage{subfigure}
\usepackage{bookmark}
% \usepackage{biblatex}
\usepackage{float}
\usepackage{url}
\usepackage{bm}% bold math
\usepackage{hyperref}% add hypertext capabilities
\usepackage[mathlines]{lineno}% Enable numbering of text and display math

\begin{document}

\title{Vicissitudes of Cities driven by Redistributive Growth}
\author{Gezhi Xiu, Jianying Wang, Lei Dong}
\author{Yu Liu}
\email{liuyu@urban.pku.edu.cn}
\affiliation{Institute of Remote Sensing and Geographic Information Systems (IRSGIS), Peking University}
\date{\today}

\begin{abstract}
    Empirical evidence suggests that the evolution of urban systems is not only determined by local conditions, but also is constrained by regional status. We propose an out-of-equilibrium model of emerging cities within a given region, which explains the spatial transitions of developmental focus and urban shrinkage phenomenon in developed cities. Meanwhile the model analytically keeps the classical results such as Clark's law for urban population density, and Zipf's law for cities' rank size distributions. We derive that these classical properties are valid for developing areas, or equivalently, most of the present cities; and the second phase of our model predicts the inevitability of various urban diseases given the limited regional resource. 
\end{abstract}
\maketitle
\section{Introduction}

Modelling urban growth dynamics through spatial models have drawn tons of concerns in the science of cites. Various significant properties of urban systems, namely Zipf's law for city's size distributions, or Clark's law for the spatial distribution of population within a city, have been addressed by models associated with promising underlying mechanisms. With such models we can further predict what equilibrium an urban systems will reach with unexpected forces such as conflagrations or pandemics. 

% 前面加综述
Despite advances in growth modelling of cities, we still lack a consistent model for both stages of demographic growth and allocation of resources.


\section{Discussion}

% Future endeavor

In this letter, we have proposed a simple mechanism, spatial preferential growth with finite seeds, to simulate the emergence of cities and reveal the scaling laws as well as how economical dilemma would lead to spatial transitions. We account the emergence and growth of cities by adding and redistributing active population over given area. The model reveals the competition among cities in area and developmental potentials. We 
This model leads a way in the adaptation of realistic conditions in statistical physical modelling, by regardless of the whole present population within the system, and considering only the active part of them. 





\end{document}