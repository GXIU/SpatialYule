\documentclass[aps,prl]{revtex4-1}
\usepackage{graphicx,amssymb,amsmath}
\bibliographystyle{apsrev4-1}
\renewcommand*{\citenumfont}[1]{S#1}
\renewcommand*{\bibnumfmt}[1]{[S#1]}


\begin{document}


\title{Supplementary Material\\Memory Matters for Cities}
\date{\today}
\author{Gezhi Xiu, Jianying Wang, Lei Dong}
% \email{xiugz@pku.edu.cn}
% \affiliation{IRSGIS, Peking University}

\author{Yu Liu}
\email{liuyu@urban.pku.edu.cn}
\affiliation{IRSGIS, Peking University, Beijing, China}
% \altaffiliation{CAMS (CNRS/EHESS) 190-198, avenue de France, 75244 Paris Cedex 13, France}

\pacs{} 

% If your reference list includes text notes as well as references,
% include the following line; otherwise, comment it out.


\maketitle
\tableofcontents
In this Supplementary Material, we provide details on ideas of model formulation, methodology, proofs, and empirical tests for the Letter Memory Matters for Cities.
\section{Details on the simulations}

The simulation results presented here are obtained in the following way. Instead of conducting the designed protocol, we do it in a equivalent way by stretching timeline to events labeled in integer. At each time step, we first decide if we add a new city, with probability $p(S)$, or a new meta-population to the existing city, with probability $1-p(S)$. The probability $p(S)$ is determined by the total 

\section{Phase one: free growth phase}

\subsection{derivation for Zipf's law of urban rank sizes}

\subsection{proof for Clark's law}

\subsection{Numerical verifications}

\section{Spatial coherence}

\section{Relative relationship between urban memory and urban size}

We give a numerical tests for this discussion. 

\section{phase two: resource restrictions}

\subsection{superior switching}

\subsection{urban shrinkage}




\begin{thebibliography}{99}
    % \bibitem{David:1970} H.A. David, H.N. Nagaraja, {\it Order statistics}  (John Wiley \& Sons, Inc., 1970).
    % \bibitem{Massey:1951} F.J. Massey, {\it Journal of the American statistical Association} {\bf 46}, 68-78 (1951).
\end{thebibliography}
    
    
\end{document} 
    