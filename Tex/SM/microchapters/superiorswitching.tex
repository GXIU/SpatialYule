Regional development is complex and is affected by differnet determinants in different time. For example, in Hebei province of China, the largest city Shijiazhuang started to develop fast since it is the crossroad of railways. Jilin City in Jilin province used to be the capital of Jilin province until Changchun, from 1954, replaced its status. Obviously, these are redistributions of social resource. In many literature, social injustice is strengthened by the techniacal revolutions. Stately, agricultural society is more likely to store fortune overtime than nomadic society\cite{doi:10.1086/701789}. The Gini coefficients in ancient society is also increasing as the productivity grows\cite{kohler2017greater}. We read this as inequality of human society is born with the desire of better life. 

But what is more interesting is that, when doesn't the rich get richer? In this Letter we give a possible explanation that the \emph{memory} of human society is the force of future endeavour. A city's future ability to develop is based on the active population brought by the existing active people. The limited resource is dynamically divided by increasing number of cities, so that there is, though small, but still some odd of overturn. In the following text, we first derive an approximated solution of this problem. We denote the coins share in the $k$th largest city as $n_k$, and investigate the probability that the second largest city becoming the largest, $P_{\text{overturn}}$. In $m$ successive additions of meta-population from the moment $t_l$, the chance for $n_2(t_l+m)\ge n_1(t_l+m)$ is equivalent to the case that the city $2$ is picked $n_1(t_l)-n_2(t_l)$ times more than city 1.

\bibliography{ref}